\documentclass{article}
\usepackage[a4paper, total={7in, 8in}]{geometry}
\usepackage{amsmath}
\usepackage{graphicx}
\usepackage{hyperref}
\usepackage{fancyvrb,xcolor}
\usepackage[latin1]{inputenc}

\title{CS212: Assignment 9}
\author{Md Shabbir Jamal}
\date{Department of Computer Science and Engineering\\
BIT, Mesra, Ranchi\\
btech10026.20@bitmesra.ac.in
}

\begin{document}
\maketitle

\begin{enumerate}
    
\item {{\large Write a program to implement Optimalpage replacement algorithm. Find the number of page faults for the following reference string:0, 2, 1, 6, 4, 0, 1, 0, 3, 1, 2, 1Verify the above reference string for 3,4 and 5 number of page frames in memory.}
\begin{verbatim}
    #include<bits/stdc++.h>
    using namespace std;
    
    int main()
    {
        /* n - number of reference elements
           frames - number of frames in memory
           fault - number of page faults
           hit - number of page hits
           front - keep the "first in" element's index
        */
        int n, frames, fault = 0,hit = 0,front = 0;
    
        cout<<"Enter reference string size : ";
        cin>>n;
    
        //  ref_s - stores reference string
        int ref_s[n];
        cout<<"Enter reference string : ";
        for(int i = 0;i<n;i++)
        {
            cin>>ref_s[i];
        }
        cout<<"Enter number of page frames : ";
        cin>>frames;
    
        //table - its to show memory status
        vector<vector<int>> table(frames);
        for(int i = 0;i<frames;i++)
        {
            table[i] = vector<int>(n,-1);
        }
    
        // cur_mem - stores current position of memory
        vector<int> cur_mem(frames,-1);
    
        //inlist - it shows if an element was already present in memory or not
        bool inlist = false;
        for(int i = 0;i<n;i++)
        {
            map<int,int> help;
            inlist = false;
            for(int j = 0;j<frames;j++)
            {
                if(cur_mem[j] == ref_s[i])
                {
                    hit++;
                    inlist = true;
                    break;
                }
                if(cur_mem[j] == -1)
                {
                    fault++;
                    cur_mem[j] = ref_s[i];
                    inlist = true;
                    break;
                }
            }
            if(inlist == false)
            {
                fault++;
                int pt = -1,jpt = 0;
                for(int j = 0;j<frames;j++)
                {
                    for(int k = i+1;k<n;k++)
                    {
                        if(cur_mem[j] == ref_s[k])
                        {
                            if(pt < k)
                            {
                                pt = k;
                                jpt = j;
                            }
                            break;
                        }
                        if(k == n-1)
                        {
                            jpt = j;
                            j = frames;
                            break;
                        }
                    }
                }
                cur_mem[jpt] = ref_s[i];
            }
            for(int j = 0;j<frames;j++)
            {
                table[j][i] = cur_mem[j];
            }
        }
    
    
        // X - in the ouput means that frame is empty
        cout<<"\nref. str ";
        for(int i = 0;i<n;i++)
        {
            cout<<ref_s[i]<<" ";
        }
        cout<<"\n\n";
        for(int i = 0; i<frames;i++)
        {
            cout<<"Frames : ";
            for(int j =  0;j<n;j++)
            {
                if(table[i][j] == -1)
                {
                    cout<<"X"<<" ";
                }
                else
                {
                    cout<<table[i][j]<<" ";
                }
            }
            cout<<endl;
        }
    
        //Result
        cout<<"\tResult"<<endl;
        cout<<"\t\tFaults : "<<fault<<endl;
        cout<<"\t\tHits   : "<<hit<<endl;
    
        return 0;
    }
\end{verbatim}
\textbf{Output}
\begin{verbatim}
    
    1.) Frame = 3

    Enter reference string size : 12
    Enter reference string : 0 2 1 6 4 0 1 0 3 1 2 1
    Enter number of page frames : 3
    
    ref. str 0 2 1 6 4 0 1 0 3 1 2 1
    
    Frames : 0 0 0 0 0 0 0 0 3 3 2 2
    Frames : X 2 2 6 4 4 4 4 4 4 4 4
    Frames : X X 1 1 1 1 1 1 1 1 1 1
            Result
                    Faults : 7
                    Hits   : 5

    2.) Frame = 4

    Enter reference string size : 12
    Enter reference string : 0 2 1 6 4 0 1 0 3 1 2 1
    Enter number of page frames : 4
    
    ref. str 0 2 1 6 4 0 1 0 3 1 2 1
    
    Frames : 0 0 0 0 0 0 0 0 3 3 3 3
    Frames : X 2 2 2 2 2 2 2 2 2 2 2
    Frames : X X 1 1 1 1 1 1 1 1 1 1
    Frames : X X X 6 4 4 4 4 4 4 4 4
            Result
                    Faults : 6
                    Hits   : 6  
    
    3.) Frame = 5

    Enter reference string size : 12
    Enter reference string : 0 2 1 6 4 0 1 0 3 1 2 1
    Enter number of page frames : 5
    
    ref. str 0 2 1 6 4 0 1 0 3 1 2 1
    
    Frames : 0 0 0 0 0 0 0 0 3 3 3 3 
    Frames : X 2 2 2 2 2 2 2 2 2 2 2
    Frames : X X 1 1 1 1 1 1 1 1 1 1
    Frames : X X X 6 6 6 6 6 6 6 6 6
    Frames : X X X X 4 4 4 4 4 4 4 4
            Result
                    Faults : 6
                    Hits   : 6  

\end{verbatim}
}

\end{enumerate}
\end{document}
